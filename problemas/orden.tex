\begin{center}
\Large \textbf{Problema B} \\
\say{\texttt{orden}}
\end{center}

Un cuadrado perfecto es un número que se puede expresar como el cuadrado de un entero. Por ejemplo, 16 es un cuadrado porque $16 = 4 \times 4$.

\bigskip
Nos interesa, dado un conjunto de números, ordenarlos de tal manera que la suma de cada dos números consecutivos sea un cuadrado perfecto.
Por ejemplo, si el conjunto es $\{2, 7, 9, 14\}$, un órden válido sería $[9, 7, 2, 14]$, ya que:
\begin{itemize}
    \item $9 + 7 = 16 = 4 \times 4$
    \item $7 + 2 = 9 = 3 \times 3$
    \item $2 + 14 = 16 = 4 \times 4$
\end{itemize}

\underline{Dar un orden del conjunto $\{1, 3, 6, 8, 10, 13, 15\}$,} \\
\underline{tal que la suma de cada dos números consecutivos sea un cuadrado perfecto.}

