\begin{center}
\Large \textbf{Problema E} \\
\say{\texttt{ping pong}}
\end{center}

Ana y Beto juegan regularmente partidos de ping-pong.
Un partido de ping-pong consiste de una serie de sets, y se juega al mejor 5 sets - el primer jugador en ganar 3 sets gana el partido.
Un set es ganado por el jugador que primero alcanza 4 puntos y tiene una ventaja de al menos 2 puntos sobre su oponente.
Es decir, si ambos jugadores tienen al menos 3 puntos cada uno (conocido como \say{deuce}), entonces se necesita ganar dos puntos consecutivos para ganar el juego:
uno para obtener ventaja y otro para ganar el set.

\bigskip
Ana y Beto jugaron varios partidos y para cada partido que jugaron anotaron una secuencia de letras, donde \lq A' representa un punto ganado por Ana y \lq B'
representa un punto ganado por Beto. Luego de jugar varios partidos arman otra secuencia de letras - una \lq A' para cada partido ganado por Ana y una \lq B' para
cada partido ganado por Beto. Por ejemplo, para tres partidos que jugaron anotaron las siguientes secuencias (agregamos \texttt{()} y \texttt{[]} para indicar
visualmente quién ganó cada set).

\begin{verbatim}
    1° partido: (AAAA)[BBBB](ABAABBAA)(AAABBA)
    2° partido: (BAABAA)[BBBB][ABBBB][BBBAAB]
    3° partido: (BAABABABAA)[ABBABB](BAABABAA)[ABBABABABB](BBAAAA)
    Resultado:  ABA
\end{verbatim}

El resultado es \texttt{ABA} porque los ganadores de los partidos fueron, respectivamente, Ana (ganó 3-1 en sets en el 1° partido),
Beto (ganó 3-1 en sets en el 2° partido) y Ana de nuevo (ganó 3-2 en sets en el 3° partido).

\bigskip

Dados los siguientes partidos jugados:
\begin{verbatim}
    1° partido: BABBAABBABBABABABBABBABABB
    2° partido: AABBBBBAAAAABAABBAAAAABBA
    3° partido: BAABBAABBAAABAABBAABBABBABBAABBAABBBABBAABBAABAABAABBAABBAAA
    4° partido: BAABAABAABAABBAAAA
    5° partido: BBBAABBAABABABAABBBBABBABABABB
    6° partido: BAABBAABBABBAAABBAABBAABBAABBBBAABBAABBAAAABBAABBAABAA
    7° partido: BBBAABBAABBAABBAAAABBABABBAAABBABBBAB
    8° partido: ABBAABBAABBBBAABBAABBABBABBAABBAABAABAABBAABBAAAABBABABB
    9° partido: BBAAAABBBABBAABABAAABBBBABBABB
    Resultado:  ???
\end{verbatim}

\underline{¿Cuál es el resultado final que anotarían?}

% Sets de ejemplo:

% Gana Ana:
% AAAA       (4-0)
% AAABA      (4-1)
% BAAAA      (4-1)
% AAABBA     (4-2)
% BBAAAA     (4-2)
% BAABAA     (4-2)
% ABAABBAA   (5-3)
% BAABABAA   (5-3)
% BAABABABAA (6-4)
% ABBAABBAABAA (7-5)
% BAABBAABBAAA (7-5)

% Gana Beto:
% BBBB       (4-0)
% BBBAB      (4-1)
% ABBBB      (4-1)
% BBBAAB     (4-2)
% AABBBB     (4-2)
% ABBABB     (4-2)
% BABBAABB   (5-3)
% ABBABABB   (5-3)
% ABBABABABB (6-4)
% BAABBAABBABB (7-5)
% ABBAABBAABBB (7-5)